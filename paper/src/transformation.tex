\section{Transformation d'un modèle à l'autre}

	Nous développons ici les étapes de compilation necessaire pour passer d'un modèle de programmation classique utilisant Javascript, vers le modèle d'execution fluxionnel décrit dans la précédente section.

	Nous listons ici l'ensemble des points de différence entre les deux modèles nécessitant une transformation lors de la compilation.

	Nous considérons qu'un service web se situe dans un sous-ensemble des programmes classiques écrit dans un langage dynamique.
	Ce sous-ensemble implique que le programme soit structuré de manière à enchaîner des traitements séquentiellement les uns après les autres.

	\subsection{Différences entre les deux modèles}

		\subsubsection{Utilisation de la mémoire}

			Dans le modèle classique, la mémoire est centrale.
			Elle est cependant cloisonné en scopes de fonctions, et en contexte d'exécution.
			Une fonction n'as accès qu'à son scope, et au contexte dans lequel elle est exécutée.

			Dans le modèle fluxionnel, la mémoire est décentralisé et encapsulé dans des scopes.

			1. Une fonction n'utilise aucune mémoire, ou n'utilise aucune persistance entre ses invocations.
			    -> pas de scope

			\begin{code}[Javascript, caption={Code fluxionnel},label={lst:mem1}]
				function(mon_argument) {
				    return mon_argument + 3; // traitements
				}
			\end{code}


			2. Une fonction utilise une closure comme scope et elle est la seule à accéder à cette closure.
			    -> scope représentant la closure

			\begin{code}[Javascript, caption={Code fluxionnel},label={lst:mem2}]
				function(mon_scope) {
				    // mon_scope accessible
				    return function(mon_argument) {
				        // mon_argument et mon_scope accessible
				        return mon_scope + mon_argument; // traitements
				    }
				}
			\end{code}

			3. Une fonction utilise une closure comme scope et elle n'est pas la seule à accéder à cette closure, cas de l'objet global.
			    -> scope représentant la closure, partagé entre les fluxions.

			\begin{code}[Javascript, caption={Code fluxionnel},label={lst:mem3}]
				var global ...

				function fn1(mon_scope) {
				    // global, mon_argument accessibles
				    return mon_scope + global; // traitements
				}

				function fn2(mon_scope) {
				    // global, mon_argument accessibles
				    return mon_scope + global; // traitements
				}
			\end{code}

		\subsubsection{Appel de fonction}

			Dans le modèle classique d'un service web, les fonctions de traitement sont appelé les unes après les autres en suivant un principe de chaîne de traitement.

			Dans le modèle fluxionnel, le pointeur d'exécution est passé de manière événementiel, porté par le système de messagerie.

			1. Une fonction appelle une autre fonction à la fin de son exécution
			    -> la fluxion représentant la fonction appelante envoie un message à la fluxion représentant la fonction appelée.

			\begin{code}[Javascript, caption={Code fluxionnel},label={lst:fn1}]
				function(req) {
				    // traitements sur req
				    return next(req);
				}
			\end{code}

			fluxion appelante -> fluxion appelée

			2. Une fonction appelle une autre fonction avant la fin de son exécution.
			    -> la fonction appelante est découpé en deux fluxions.
			    la fluxion représentant la fonction appelée va servir d'intermédiaire entre ces deux fluxions.

			fluxion appelante 1 -> fluxion appelée -> fluxion appelante 2
