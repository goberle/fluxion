\section{Modèle d'exécution fluxionnel}

	\subsection{Fluxions}

		Le principe de modèle d'exécution fluxionnel est d'identifier des unités d'exécution autonomes fondés sur des flux.
		Une unité est autonome quand elle peut être déplacé dynamiquement d'environnement d'exécution pendant sont activité.
		Déplacer une unité d'exécution nécessite de déplacer avec le code d'exécution son contexte courant. C'est à dire l'ensemble des variables d'état et de mémoire provenant des précédentes exécutions de l'unité.
		Dans notre approche, ce contexte est encapsulé sous forme de flux pour être manipulé par l'unité d'exécution.
		(Pour notre approche nous "transférons" ce contexte dans un flux propre à l'unité déplacé.)
		Ainsi, déplacer une telle unité consiste à déplacer le code fonctionnel vers une nouvelle destination, puis à rediriger les flux d'entrées et de sorties en conséquence. Une telle unité peut alors être déplacé de nœud en nœud sans y être supprimée (attachée?).
		Seul les flux ont besoin d'être redirigés en conséquence.

		Nous avons appelé cette unité d'exécution autonome une fluxion. C'est à dire une fonction, au sens de la programmation fonctionnelle ne dépendant pour ses entrées et ne produisant sur ses sorties que des flux.

		\subsubsection{Persistance des états}

			Les fluxions n'ont pas d'état afin de pouvoir être transporté d'un nœud à l'autre en cours d'exécution. Comme de nombreuses applications reposent sur l'existence d'un tel état, nous l'avons transféré dans les flux d'entrées et de sortie des fluxions. Ainsi l'état suit la mobilité d'une fluxion lors de la redirection de ses flux.
			// TODO est-ce qu'on laisse cette explication sur l'état volontairement plus simpliste, ou est-ce qu'on explique le détail technique de `apply` ?
			L'état d'une fluxion est ainsi modélisé par un flux d'entré supplémentaire appelé scope. Lorsqu'une fluxion a fini de s'exécuter elle envoie dans son flux scope les données qu'elle souhaite persister, et qui seront re-agregé à son prochain appel.

		\subsubsection{Adressage et mobilité des fluxions}

			Le système de messagerie distribue les messages aux fluxions en se basant sur l'adresse de destination du message.
			Chaque fluxion est associé à une adresse permettant au système de messagerie de l'identifier de manière unique.

			Lorsqu'une fluxion se déplace d'un nœud à l'autre, son adresse se déplace aussi, et ce avec les flux associé à cette adresse.
			Les deux nœuds communiquent ce déplacement entre eux pour le valider, et auprès des autres nœuds pour les en informer.
			De cette manière chacun des nœuds connaît l'emplacement de chacune des fluxions sur les autres nœuds.
			Ainsi le système de messagerie composé de l'ensemble de ces nœuds peut acheminer n'importe quel message sur l'ensemble du système.

		\subsubsection{entrées \/ sorties}

	\subsection{Cycle de vie}

		Une application fluxionnelle est composée d'un enchaînement de fluxions.
		Chaque fluxion présente le même comportement : 
		\begin{itemize}
			\item elle est invoqué par un système de messagerie à la réception d'un message,
			\item effectue des opérations à partir du message reçu,
			\item modifie son état interne déporté dans le système de messagerie sous forme de scope,
			\item  puis renvoie un message.
		\end{itemize}

		Dans notre approche, un message est une structure de couples clé / valeur  contenant deux couples : le nom de la fluxion à invoquer `addr` et le corps du message `body`.
		Le système de messagerie impose deux fonctions :
		\begin{itemize}
			\item une fonction d'enregistrement
		    \lstinline|register(<nom>, <fn>, <contexte>)| et
			\item une fonction de déclenchement de la chaîne de traitement
		    \lstinline|start(<nom>,<param>)|.
		\end{itemize}


		Les données et la logique d'une application étant cloisonné de manière distincte pendant l'exécution, il est possible de mettre à jour une fluxion en la remplaçant dans le système, sans impacter l'exécution de l'application.

		(Comme les données propres au fonctionnement d'une application sont stockées dans le système de messagerie et que les fluxions ne possèdent pas de données propres, l'installation d'une nouvelle version se fait en enregistrant en cours d'exécution les nouvelles fluxions.)

		De plus la relocalisation d'une fluxion se fait de manière transparente par l'application, par le système de messagerie qui connaît la localisation exacte des fluxions. Nous y reviendrons plus tard, mais la relocalisation d'une fluxion consiste à déplacer les fluxions sur un nouveau nœud et de rediriger les messages en conséquences. Comme les types de messages, leur débit et le contexte propre d'une fluxion sont connus, on connaît a priori le coût de migration à chaud d'une fluxion.

	\subsection{Architecture Web}

		Le système fluxionnel ne manipule que des fluxion par l'intermédiaire d'un système de messagerie. Afin de pouvoir interagir avec le monde extérieur, il faut définir des interfaces de bordure. Notre approche repose sur une espérance de gain technologique principalement sur les architectures Web. Le premier point d'entré visé est ...(?) les interfaces REST.
		Le schema suivant présente la séparation du système en fonctions indépendantes.

		\TODO{schema}

		Le système Web est donc le déclencheur d'une chaîne de traitement de requêtes à chaque nouvelle requête d'un utilisateur un appel à la fonction \lstinline|start('/', <param>)| est réalisé dans le système de messagerie.
		Au démarrage du système Web, deux demi-fluxions sont lancées.
		La demi-fluxion 'in' n'est pas enregistré dans le système de messagerie.
		Elle prend les paramètres de la requête Web, place l'identifiant de la connexion client dans le contexte de la demi-fluxion de sortie, puis lance le traitement de la requête en invoquant la fonction `start` du système de messagerie.

