\section{Introduction}


The growth of web platforms is caused partially by the capacity of Internet to favor services development with the quick release of a minimal product.
In a matter of hours, it's possible to put online a working product to gather a first community around.
\textit{``Release early, release often''} is often heard to quickly gather a user community around an open source project, as a community is a factor of success.

If the service complies with requirements, the community will probably grow gradually as the service gain popularity.
To absorb this growth, the resources quantity used by the service will grow accordingly.
Until the size of data to process, and available resources require to use a more efficient processing model.
To reduce coupling between service's parts, and migrate them to more resourceful environment, most of the most efficient models split the exchanges between functions, using different communication paradigms like three-tiers architecture, events, messages, or stream. [biblio]
Once split, the different service's parts are connected by a messaging system, often asynchronous.
Lots of tools have been defined to express and manage these different service's parts and their communications [Storm, MillWheel, Spark, TimeStream ...]
However these tools use specific interfaces and languages.
It requires to train the development teams, hire experts and write again the initial code base.
This new architecture is mainly less flexible and less adaptable to quick modification.
Thus, this modifications implies to take risk to continue developing the project because of the implications, without releasing a concrete impact on the service's features.

We propose a tool to automate this architecture shift, by using a split architecture to the program, without having to modify the initial code base.
Such a tool could lift the risks described above.
We aim at providing this tool to Web applications of which load come from users requests streams, and initial development use a simple web paradigm.
% TODO explain a simple web paradigm
We think it's possible to analyze this type of applications to express it using autonomous, movable functions communicating by data streams, as soon as the first public release.

We assume this applications are developed in a dynamic language like Javascript, and we propose a tool able to identify internal streams, to define stream processing units, and to dynamically manage these units.
The tool aim to not modify the existing code, but propose a layer of meta information over the initial code.
This layer uses the paradigm of fluxion which we are going to define, and will be at the core of our proposition.

% La croissance des plateformes du Web est dû partiellement à la capacité d'Internet à favoriser le développement de services avec une mise en production minimale très rapide.
% En quelques heures, il est possible de mettre en ligne un produit fonctionnel afin de rassembler une première audience.
% \textit{``Release early, release often''} est souvent entendu pour capter rapidement une communauté d'utilisateurs autour d'un projet open source.

% Si le service répond correctement aux attentes, l'audience va probablement grossir au fur et à mesure que le service gagne en popularité.
% Pour pouvoir faire face à cette croissance, la quantité de ressources utilisé par le service augmente en conséquence, et il arrive un moment dans le développement du produit où la taille des données à traiter et la quantité de ressources nécessaires, imposent l'utilisation d'un modèle de traitement plus efficace.
% La plupart des modèles plus efficaces passent par une segmentation des échanges entre fonctions, en utilisant différents paradigmes de communication comme les approches \textit{three-tiers}, les événements, les messages ou les flux, afin de réduire le couplage entre les parties et pouvoir les migrer vers des environnement de plus en plus puissants. [bilbio]
% Une fois segmenté, les différentes parties communiquent entre elles par un principe de messagerie le plus souvent asynchrone.
% De nombreux outils ont été définis qui permettent d'exprimer ces différentes parties, leurs interactions, et de prendre en charge l'acheminement des messages [Storm, MillWheel, Spark, TimeStream ...].
% Cependant, ces outils utilisent des interfaces et des langages particuliers.
% Il est nécessaire de former les équipes de développement à leur utilisation, d'engager des experts et de réécrire le code initial.
% Cette nouvelle architecture est globalement moins souple et souvent moins propice aux changements rapides, de ce fait le changement d'architecture présente une prise de risque dans la poursuite du projet de par les modifications impliqués sur le developpement, sans pour autant d'impact direct sur la nature du service rendu.
% // TODO à vérifier et documenter [biblio]

% Nous proposons un outil visant à automatiser ce changement d'architecture, en apportant une vision segmentée du programme sans modifier le code initalement développé.
% Un tel outil permettrait de lever les risques décrits ci-dessus.
% // TODO notion de lazy transformation
% Nous visons des applications Web dont les sollicitations proviennent des flux de requêtes utilisateurs et dont le développement initial est réalisé selon une approche web 'basique' (serveur web / traitement applicatif / data).
% Nous pensons qu'il est possible d'analyser cette classe d'applications dès les premières étapes d'exploitation afin de les exprimer sous la forme de flux échangés entre fonctions autonomes, relocalisables.

% Nous supposons que les applications sont développées dans un langage dynamique comme Javascript, et nous proposons un outil capable d'identifier les flux internes d'échanges, de définir des unités de traitement de ces flux, et de pouvoir gérer de manière dynamique ces unités.
% L'outil identifie ces unités sans être intrusif dans le code existant mais en proposant une sur-expression du programme initial reposant sur le paradigme de fluxion que nous allons définir et qui servira de cœur à notre proposition.

\TODO{La section 2 présente le principe de fluxion en le positionnant par rapport à l'existant.
La section 3 ...}
