\section{Introduction}


The growth of web platforms is caused partially by Internet's capacity to stimulate services development and allow very quick release of minimal viable products.
In a matter of hours, it's possible to upload a first product and start gathering a community around.
\textit{``Release early, release often''} is commonly heard as an advice to quickly gather a user community, as the size of the community is a factor of success.

If the service complies successfully with users requirements, the community will grow gradually as the service gain popularity.
To cope with this growth, the resources quantity taken up by the service shall grow exponentially.
Until the amount of data to process requires the development team to use a more efficient processing model to make better use of the resources.
Many of the most efficient models split the system into parts to reduce coupling and migrate them to more resourceful environment. [biblio]
Once split, the different service's parts are connected by a messaging system, often asynchronous, using communication paradigms like \textit{three-tiers} architecture, events, messages or stream.
Many tools have been developed to express and manage these different service's parts and their communications [Storm, MillWheel, Spark, TimeStream ...]
However these tools use specific interfaces and languages.
Thus, it requires the development team to be trained, to hire experts and to start over the initial code base, while this new architecture is not flexible and adaptable enough for quick modifications, as the initial code base was.
Thus, this modifications implies the development team to take risks without adding concrete value to the service.

We propose a tool able to automate this architecture shift.
Such a tool might lift the risks described above.
We aim at providing this tool to Web applications for which load come from users requests streams.
And application for which initial development use a simple web paradigm consisting of a web server, data processing logic, and a database.
We think it's possible to analyze this type of applications to express it using autonomous, movable functions communicating by data streams.
And to shift architecture as soon as the first public release without wiping off the initial code base.

We assume these applications are developed in a dynamic language like Javascript, and we propose a tool able to identify internal streams and stream processing units, and to dynamically manage these units.
The tool aims not to modify the existing code, but propose a layer of meta information over the initial code.
This layer uses the paradigm of fluxion which we are going to define in section 2, and will be at the core of our proposition of automation, described section 3.
In Section 4, we evaluate our tool, and compare it to other solutions in term of performance, and development impact.